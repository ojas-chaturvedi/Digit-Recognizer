\documentclass{article}
\usepackage[utf8]{inputenc}
\usepackage{amsmath}
\usepackage{amssymb}
\usepackage{xcolor} % for color definitions
\usepackage{mdframed} % for framing and shading the problems
\usepackage{lipsum} % for generating text, remove in your actual document
\usepackage{geometry} % for page layout
\usepackage{titling} % for title page layout

% Set the page size and margins
\geometry{letterpaper, portrait, margin=1in}
\setlength{\droptitle}{-1in}

% Define a new environment for the problems that takes one argument for the problem number
\newenvironment{problem}[1]{
    \begin{mdframed}[backgroundcolor=gray!20, skipabove=\baselineskip, skipbelow=\baselineskip, nobreak=true, innerleftmargin=10pt, innerrightmargin=10pt, innertopmargin=10pt, innerbottommargin=10pt]
    \textbf{Problem #1.}
}{
    \end{mdframed}
}

% Define a new environment for the proofs
\newenvironment{proof}{
    \begin{mdframed}[nobreak=true, innerleftmargin=10pt, innerrightmargin=10pt, innertopmargin=10pt, innerbottommargin=10pt]
    \textbf{Proof.}
}{
    \hfill $\square$
    \end{mdframed}
}

% Remove section numbering
\makeatletter
\renewcommand{\@seccntformat}[1]{}
\makeatother

% Remove table of contents numbering
\renewcommand{\thesection}{}

% Title page info
\title{AI-Based Precision Medicine Platform -- Proposal \\ \large Capstone: The Art of Approximation}
\author{Ojas Chaturvedi, Zaheen Jamil, Saianshul Vishnubhaktula, Ritwik Jayaraman}
\date{November 21, 2023}

% ------------------------------------------------------------------------------
\begin{document}

% % Title page
% \begin{titlingpage}
% \end{titlingpage}
\maketitle

% ------------------------------------------------------------------------------

\begin{enumerate}
    \item \textbf{Language:} Python, a simple and popular language for machine learning and data science due to its extensive libraries and frameworks
    \item \textbf{Objective:} To build an AI-powered platform to analyze symptoms, previous medical records, and research to provide chances of diseases and then give personalized treatment recommendations
    \item \textbf{Implementation:}
        \begin{enumerate}
            \item \textbf{Overview of steps:}
                \begin{enumerate}
                    \item \textbf{Data Collection:} Collect previous medical records and research data from various sources and databases
                    \begin{enumerate}
                        \item Ex: github.com/bruzwen/ddxplus
                    \end{enumerate}
                    \item \textbf{Data Processing:} Process the data to extract relevant features, such as removing stop works and blank lines, etc.
                    \begin{enumerate}
                        \item \textbf{Homomorphic Encryption} will protect sensitive health data instead of conforming to HIPAA and other health data protection regulations
                    \end{enumerate}
                    \item \textbf{Model Training:} Train machine learning models to predict disease risk and treatment outcomes
                    \item \textbf{Model Deployment:} Deploy the models on a secure platform to be used by clinicians and patients
                \end{enumerate}
            \item \textbf{Libraries:}
            \begin{enumerate}
                \item \textbf{Pandas:} For data manipulation and analysis
                \item \textbf{Matplotlib:} For visualizations
                \item \textbf{NumPy:} For numerical computing and working with arrays
                \item \textbf{Scikit-learn:} For data mining and analysis
                \item \textbf{TensorFlow:} For complex neural network modeling
                \item \textbf{PyTorch:} For natural language processing
                \item \textbf{NLTK/spaCy:} For human language data with symptom inputs
                \item \textbf{Flask/Django:} For backend web development
                \item \textbf{SQLAlchemy:} For SQL databases and Object-Relational Mapping
            \end{enumerate}
            \item \textbf{Manual Work:}
                \begin{enumerate}
                    \item Building the machine learning model
                    \item Collection of datasets of diseases and percent chance of symptoms
                    \begin{enumerate}
                        \item Will contact local hospitals for datasets for the latest outcomes
                    \end{enumerate}
                    \item Homomorphic encryption implementation
                    \item GUI development
                    \begin{enumerate}
                        \item Website or App
                        \item Users can input their symptoms (and this will go into the dataset anonymously with homomorphic encryption)
                        \item Users receive a percent chance of diagnosis based on these symptoms from the model we trained \\ \\
                    \end{enumerate}
                \end{enumerate}
        \end{enumerate}
    \item \textbf{Jobs:}
    \begin{enumerate}
        \item 2 Machine Learning/Data Collections Specialist
        \begin{enumerate}
            \item Make the Machine Learning/AI Model using Python and related libraries
            \item Compile all data needed
        \end{enumerate}
        \item Data Security Specialist
        \begin{enumerate}
            \item Focuses on the implementation of homomorphic encryption
        \end{enumerate}
        \item GUI Developer
        \begin{enumerate}
            \item Makes the website/app and all of its functionality (UI)
            \item Would work with a Data Security Expert for the implementation of homomorphic encryption
        \end{enumerate}
    \end{enumerate}
\end{enumerate}
% ------------------------------------------------------------------------------
\end{document}
% ------------------------------------------------------------------------------